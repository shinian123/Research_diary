\documentclass[2pt,a4paper]{article}
\usepackage{fontspec, xunicode, xltxtra}
\XeTeXlinebreaklocale "zh"
\XeTeXlinebreakskip = 0pt plus 1pt minus 0.1pt
\setmainfont{SimSun}
\title{Some insights from ME-TRPO}
\begin{document}
\section{ME-TRPO}
一个agent,在黑暗中,对周围的state space一无所知,只能通过try-and-error去获取生存策略(policy),了解环境(environment)。它有两种方式,第一种方式不对环境进行研究,只研究policy和state的关系。它通过每次的尝试,来总结在某一个state,如果采取相应的动作($a$),将来会得到的总收益($q(s,a)$)当然这个估计一开始是不准确的,因此agent需要每次动态刷新这个收益值。每次刷新之后,agent就会根据当前的值函数去选取最优策略,即选取使值函数最大的action。这是model-free的方法。\\
还有一种方式需要对环境进行研究,即agent先对发现它在当前的$s_0$,如果采取某一动作$a_1$,就会转移到$s_1$,而若采取动作$a_2$,就会转移到另一个状态$s_2$,它发现这其中有个对应关系。于是它转而先研究这个对应关系,然后根据研究的结果来选择策略。如果它的起始状态是$s_0$,而它想到$s_g$,则agent会在simulator中构造出一种从$s_0$到$s_g$的路径。agent不需要亲自到$s_g$,而是根据采样的轨迹就可以推知$s_t$,$a_t$,$s_{t+1}$的关系。这就是model-based的方法。这种方法的缺点在于模型和实际世界有很大的偏差。在这篇文章中,作者认为model bias的原因在于agent在policy optimization的过程中exploit了数据不够充足以训练的region。因此,作者提出使用深度网络元组来训练model,以保持model的uncertainty。目前搞不明白的是为什么这样可以让agent避免上述那点原因。\\\\
model-based的特性目前还不是很清楚。不过可以思考下面几个问题:\\
1.为什么vanilla model-based RL会在policy optimization阶段exploit数据并不充足的区域?\\
2.model learning和policy optimization两者之间的关系是什么?\\
先来看问题2。agent先使用初始的policy去sample real trajectory,然后使用sample去fit transition function,接着利用fit的model产生synthetic trajectory,然后使用这些模拟的路径去optimize policy。整个过程循环,就是vanilla model-based RL。\\
分析这个过程,policy optimization和model learning的联系有两个:sample和synthetic trajectory。前者是实际的trajectory,由当前policy执行产生,用来train dynamics model,后者是synthetic trajectory,由dynamics model产生,用来train policy。这是一个包含四个环节的循环。这是Dyna的模型之源头。\\
分析一下误差在哪里产生。误差主要来源于prediction error。model无论如何精确都不可避免地会和实际世界有误差,而且在产生trajectory的过程中,每一步都会产生误差,并逐渐累积。这部分误差可以通过model-free的方式来减轻。这就是model-free和model-based结合的方法。\\
policy error:$d(\pi_{real},\pi_{syn})$\\
Prediction error:$\frac{1}{D}\sum_{(s_t,a_t,s_{t+1}\in{D})}\parallel{s_{t+1}-f_{\phi}(s_t,a_t)}\parallel^2$\\
令$\hat{s}_{t+1}=f_{\phi}(s_t,a_t)$\\
model learning的缺点在于对$s\in{S_{sample}}$,prediction 比较准确,而对于$s\notin{S_{sample}}$,则不一定能保证正确。这也就是所谓的overfitting,或者是在文章里作者所说的在policy optimization阶段agent在这些状态时,会导致误差。




\end{document}
